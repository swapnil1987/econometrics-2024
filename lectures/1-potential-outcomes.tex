\documentclass{beamer}
\usetheme{metropolis}
\usepackage{graphicx}
\usepackage{booktabs}
\usepackage{hyperref}

\title{Microeconometrics Module}
\subtitle{Lecture 1: Potential Outcomes}
\author{Swapnil Singh}
\date{Lietuvos Bankas and KTU \\ \href{https://github.com/swapnil1987/econometrics-2024}{\textcolor{magenta}{Course Link}}}

\begin{document}
	
	\maketitle
	
	\begin{frame}{Introduction}
		\begin{itemize}
			\item We are interested in causality
			\begin{itemize}
				\item Does vaccine prevents polio, Covid-19, malaria?
				\item Does schooling increases earning potential of individuals?
				\item Is dense urban development conducive for economic growth?
				\item Why female labor force participation increased in the $20^{th}$ century?
			\end{itemize}
			\item \textbf{Objective} of today's lecture
			\begin{itemize}
				\item Learn about potential outcomes: notation and intuition
			\end{itemize}
			\item \textbf{Why?}
			\begin{itemize}
				\item Fundamental block to talk about causality
			\end{itemize}
		\end{itemize}
	\end{frame}
	
	\begin{frame}{An example}
		\begin{itemize}
			\item \textbf{Question:} Does health insurance access improves health outcomes
			\item \textbf{Notation:}
			\begin{itemize}
				\item $Y$ - called \textit{outcome} - denotes health of individual
				\item $D$ - called \textit{treatment} - denotes whether individual has health insurance or not
			\end{itemize}
			\item Transformation of our question:
			\begin{itemize}
				\item $D$ has an effect on $Y$?
			\end{itemize}
			\item Why this question is hard?
		\end{itemize}
	\end{frame}
	
	
	
	
	\begin{frame}{Potential Outcomes}
		\begin{itemize}
			\item We do not observe both worlds simultaneously
			\item But thinking about them helps to understand causality
			\item \textbf{Potential Outcomes}
			\begin{itemize}
				\item different states that can occur for a unit 
			\end{itemize}
			\item in our example \textit{unit} is \textit{individual}
			\item different states
			\begin{itemize}
				\item individual's health without health insurance
				\item individual's health with health insurance
			\end{itemize}
		\end{itemize}
	\end{frame}
	
	\begin{frame}{Potential Outcomes: Notation}
		\begin{itemize}
			\item Neyman-Rubin Causal model
			\item $n$ individuals 
			\item index $i$
			\item Two potential outcomes for each individual
			\begin{itemize}
				\item health when have health insurance: $Y_i(D_i=1)$ or $Y_i(1)$
				\item health when don't have health insurance: $Y_i(D_i=0)$ or $Y_i(0)$
			\end{itemize}
			\item Causal effect for $i$
				\begin{align*}
					\tau_i = Y_i(1) - Y_i(0)
				\end{align*}
			\item $\tau_i$ is not observed. Why?	
			\item \textbf{SUTVA} 
			\begin{itemize}
				\item Stable unit treatment value assignment
				\item fancy name to say there are no spillover effects
				\item individual $i$'s outcome is not affected by treatment of other units
			\end{itemize}
		\end{itemize}
	\end{frame}
	
	\begin{frame}{Potential outcomes and actual outcome}
		For individual $i$ we observe $Y_i$
		\begin{itemize}
			\item Two potential outcomes for the same individual: $Y_i(0), Y_i(1)$
			\item We can write
		\end{itemize}
		\[
		Y_i = (1-D_i)Y_i(0) +  D_i Y_i(1)
		\]
		where $D_i \in \{0,1\}$
		
		\begin{table}
			\centering
			\begin{tabular*}{0.8\linewidth}{c@{\extracolsep{\fill}}cccc}
				\toprule\toprule
				i & $Y_{i}(1)$ & $Y_{i}(0)$ & $D_{i}$ & $Y_{i}$ \\
				\midrule
				1 & 0 & 1 & 1 & 0 \\
				2 & 0 & 0 & 1 & 0 \\
				3 & 1 & 0 & 0 & 0 \\
				$\vdots$ & $\vdots$ & $\vdots$ & $\vdots$ & $\vdots$ \\
				n & 0 & 1 & 0 & 1 \\
				\bottomrule\bottomrule
			\end{tabular*}
		\end{table}
	\end{frame}
	
	\begin{frame}{Reiterating the objective}
		\begin{itemize}
			\item We want to estimate the causal effect of treatment $D$
			\item Life would be easy if observe both potential outcomes
			\item But, life is not easy
			\item Essentially, our whole objective will be to construct that missing potential outcome
			\begin{itemize}
				\item counterfactual
			\end{itemize}
			\item Essence of this module: construction of counterfactuals
		\end{itemize}
	\end{frame}
	
	\begin{frame}{Next}
		\begin{itemize}
			\item Some basic primer on terminology
			\item Identification meaning
		\end{itemize}
		
		\vfill
		
		\centering
		\Large{Questions?}
	\end{frame}
	
\end{document}
