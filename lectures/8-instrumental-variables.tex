\documentclass{beamer}
%\usetheme{metropolis}
\usetheme{default}
\usecolortheme{default} % change this to change the color scheme
\usepackage{graphicx}
\usepackage{amsthm}
\setbeamertemplate{itemize items}[circle]
\usepackage{booktabs}
\usepackage{hyperref}
\usepackage{listings}
\usepackage{tikz}
\usetikzlibrary{positioning, shapes.geometric, arrows.meta}

%\newtheorem{assumption}{Assumption}
\newtheorem{assumption}{Assumption}[section]
\colorlet{shadecolor}{gray!40}
\newtheoremstyle{plain} % Just using 'plain' or any name if you're not defining a style
{}    % Space above
{}    % Space below
{\itshape} % Body font
{}    % Indent amount
{\bfseries} % Theorem head font
{.}   % Punctuation after theorem head
{.5em} % Space after theorem head
{}    % Theorem head spec (completely empty, no predefined format)

\theoremstyle{plain}
\newtheorem*{customthm}{} % No predefined name

% Command to insert theorem with just a title
\newcommand{\customtheorem}[2]{
	\begin{customthm}
		\textbf{#1.} #2
	\end{customthm}
}

\lstset{
	language=R,
	basicstyle=\ttfamily\small,
	commentstyle=\color{black},
	keywordstyle=\color{blue},
	numbers=left,
	numberstyle=\tiny\color{red},
	stepnumber=1,
	numbersep=5pt,
	backgroundcolor=\color{shadecolor},
	showspaces=false,
	showstringspaces=false,
	showtabs=false,
	frame=single,
	tabsize=2,
	captionpos=b,
	breaklines=true,
	breakatwhitespace=false,
	title=\lstname
}

\DeclareMathOperator*{\argmax}{arg\,max}
\DeclareMathOperator*{\argmin}{arg\,min}
\newenvironment{bigitemize}{\itemize\addtolength{\itemsep}{10pt}}{\enditemize}
\newcommand\independent{\protect\mathpalette{\protect\independenT}{\perp}}
\def\independenT#1#2{\mathrel{\rlap{$#1#2$}\mkern2mu{#1#2}}}
\title{Microeconometrics Module}
\subtitle{Lecture 8: Instrumental Variables}
\author{Swapnil Singh}
\date{Lietuvos Bankas and KTU \\ \href{https://github.com/swapnil1987/econometrics-2024}{\textcolor{magenta}{Course Link}}}

\begin{document}
	
	\maketitle
	
	\begin{frame}{Selection on Observables and Unobservables}
		\textbf{Understanding the Assumptions:}
		\begin{itemize}
			\item Observables:
			\[
			(Y_{0i}, Y_{1i}) \perp D_i \mid X_i
			\]
			Assumes that all relevant variables influencing treatment assignment, \(X_i\), are observed.
			
			\item Unobservables:
			\begin{enumerate}
				\item \((Y_{0i}, Y_{1i}) \perp Z_i\)
				\item \(\operatorname{Cov}(Z_i, D_i) \neq 0\)
			\end{enumerate}
			Moves beyond observable characteristics, relying on instruments (\(Z_i\)) that affect treatment but are independent from the outcomes except through treatment.
		\end{itemize}
	\end{frame}
	
	\begin{frame}{The Goals of Causal Inference}
		\textbf{Isolating Variation:}
		\begin{itemize}
			\item We aim to separate the exogenous (good) variation in \(D_i\) from the endogenous (bad) variation that correlates with outcomes \(Y_{0i}\) and \(Y_{1i}\).
			\item \textbf{Strategy:}
			\begin{itemize}
				\item Use observables \(X_i\) to control for all observable confounding.
				\item Utilize unobservables \(Z_i\) to capture the pure effects of \(D_i\) on \(Y_i\), filtering out the confounding.
			\end{itemize}
		\end{itemize}
	\end{frame}
	
	\begin{frame}{Choosing the Right Approach}
		\textbf{Which approach better handles confounding?}
		\begin{enumerate}
			\item \textbf{Observables:} Assumes perfect knowledge of confounders—challenging and often unrealistic in non-experimental settings.
			\item \textbf{Unobservables (Instrumental Variables):} More plausible as it uses instruments to isolate exogenous variation, though it may sometimes be underpowered.
		\end{enumerate}
	\end{frame}
	
	\begin{frame}{Instrumental Variables: An Introduction}
		\textbf{Isolating Good Variation:}
		\[
		Y_i = \beta_0 + \beta_1 D_i + \varepsilon_i \tag{1}
		\]
		\begin{itemize}
			\item Instrumental Variables (IV) use tools or instruments (\(Z_i\)) to ensure that \( \operatorname{Cov}(D_i, \varepsilon_i) = 0 \).
			\item This approach allows us to estimate \(\beta_1\) consistently, avoiding the bias that occurs in ordinary least squares (OLS) when \(D_i\) is correlated with \(\varepsilon_i\).
		\end{itemize}
	\end{frame}

\begin{frame}{Directed Acyclic Graph: Instrumental Variable Approach}
	\begin{center}
		\begin{tikzpicture}[scale=0.8][node distance=2cm and 2.5cm, auto, thick]
			% Nodes
			\node[ellipse, draw=black, fill=blue!20, minimum width=2.5cm, minimum height=1cm, align=center] (Z) {Instrument\\$Z_i$};
			\node[ellipse, draw=black, fill=green!20, minimum width=2.5cm, minimum height=1cm, right=of Z, align=center] (D) {Treatment\\$D_i$};
			\node[ellipse, draw=black, fill=red!20, minimum width=2.5cm, minimum height=1cm, right=of D, align=center] (Y) {Outcome\\$Y_i$};
			\node[ellipse, draw=black, fill=gray!20, minimum width=2.5cm, minimum height=1cm, above=of D, align=center] (U) {Unobserved\\confounders\\$U_i$};
			
			% Arrows
			\draw[-Latex] (Z) -- (D) node[midway, below] {};
			\draw[-Latex] (D) -- (Y) node[midway, below] {};
			\draw[-Latex] (U) -- (D) node[midway, right] {};
			\draw[-Latex] (U) -- (Y) node[midway, right] {};
			
			% Information about independence
			\draw[dashed, Latex-Latex] (Z) to[bend left=45] node[midway, above] {independent} (U);
			\draw[dashed, Latex-Latex] (Z) to[bend right=45] node[midway, below] {independent} (Y);
		\end{tikzpicture}
	\end{center}
\end{frame}
	
	\begin{frame}{Defining a Valid Instrument}
		\textbf{Criteria for Instrument Validity:}
		\begin{enumerate}
			\item \(\operatorname{Cov}(Z_i, D_i) \neq 0\) — The instrument must influence the treatment.
			\item \(\operatorname{Cov}(Z_i, \varepsilon_i) = 0\) — The instrument must not be related to the outcome other than through the treatment.
		\end{enumerate}
		These conditions ensure that \(Z_i\) is a suitable instrument for isolating the causal effect of \(D_i\) on \(Y_i\).\\
		\vspace{1em}
		First condition is easy to check. Second condition -- known as \textbf{exclusion restriction} -- can only be reasoned through. You cannot check it. Why?
	\end{frame}
	
	\begin{frame}{Two-Stage Least Squares (2SLS)}
		\textbf{Operationalizing IV:}
		\begin{itemize}
			\item \textbf{First Stage:} Estimate the effect of \(Z_i\) on \(D_i\) (possibly with other covariates \(X_i\)), obtaining \(\widehat{D_i}\).
			\item \textbf{Second Stage:} Use \(\widehat{D_i}\) to estimate the causal effect on \(Y_i\).
		\end{itemize}
		\textbf{Consistency:}
		\[
		\hat{\beta}_\text{2SLS} = \left( D' P_Z D \right)^{-1} \left( D' P_Z Y \right)
		\]
		where \(P_Z = Z (Z'Z)^{-1} Z'\) is the projection matrix of \(Z\). This estimator is consistent for \(\beta_1\), assuming valid instruments.
	\end{frame}

\begin{frame}{IV Estimation Essentials}
	\textbf{Key Considerations in Two-Stage Least Squares (2SLS) and IV Estimation:}
	\begin{itemize}
		\item The controls $(X_i)$ must be consistent across both stages to maintain coherence in the model.
		\item Standard errors from the second stage are typically biased unless corrected for the procedures used in the first stage.
	\end{itemize}
\end{frame}

\begin{frame}{Understanding the Reduced Form}
	\textbf{The Reduced Form Equation:}
	\[
	Y_i = \pi_1 Z_i + \pi_2 X_i + u_i
	\]
	\begin{itemize}
		\item This regression links the outcome directly to the instrument and control variables.
		\item It provides a consistent estimate of the instrument's effect on the outcome, bypassing endogeneity issues.
	\end{itemize}
\end{frame}

\begin{frame}{The Value of the Reduced Form}
	\textbf{Advantages of the Reduced Form:}
	\begin{itemize}
		\item Clarifies the source of identifying variation in the model.
		\item Avoids complications associated with weak instruments.
		\item Simplifies interpretation by directly estimating the effect of the instrument on the outcome.
		\[
		\widehat{\beta}_{1}^{2SLS} = \frac{\widehat{\pi}_{1}}{\widehat{\gamma}_{1}}
		\]
		This ratio illustrates the causal effect of treatment, scaled by the influence of the instrument on treatment.
	\end{itemize}
\end{frame}

\begin{frame}{Intuitive Understanding of Reduced Form}
	\textbf{Estimating Impact via 2SLS:}
	\[
	\widehat{\beta}_{1}^{2SLS} = \frac{\text{Reduced-form estimate}}{\text{First-stage estimate}} = \frac{\widehat{\pi}_{1}}{\widehat{\gamma}_{1}}
	\]
	\begin{itemize}
		\item \(\widehat{\pi}_1\): Effect of the instrument on the outcome.
		\item \(\widehat{\gamma}_1\): Effect of the instrument on the treatment.
		\item This ratio adjusts the direct instrument effect on the outcome to reflect its indirect effect via treatment.
	\end{itemize}
\end{frame}

\begin{frame}{Reduced Form Example: Scholarship Impact on Income}
	\textbf{Scenario Analysis:}
	\begin{itemize}
		\item Suppose 50\% of lottery winners graduate due to the scholarship (\(\widehat{\gamma}_1 = 0.50\)).
		\item Reduced form estimate shows a \$5,000 income increase for winners (\(\widehat{\pi}_1 = \$5,000\)).
		\item Actual effect of graduation on income, when adjusted for the scholarship effect on graduation, is:
		\[
		\frac{\$5,000}{0.50} = \$10,000
		\]
		This calculation adjusts for the proportion of scholarship recipients who graduate, doubling the perceived income benefit.
	\end{itemize}
\end{frame}

\begin{frame}{Deep Dive into IV Mechanics}
	\textbf{Further Insights:}
	Explore the nuances of IV estimation, focusing on the subtleties and implications of instrument strength and exogeneity.
\end{frame}

	

\end{document}